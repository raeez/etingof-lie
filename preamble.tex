%%%%%%%%%%%%%%%%%%%%%%%%
%% project modularity %%
%%
\usepackage[subpreambles=true]{standalone}
\usepackage{import} % more elegant than \input for standalone documents

%%%%%%%%%%%%%%%%%%%%%%%%%%%%%%%%%%
%% design, formatting and fonts %%
%%

%\usepackage[titletoc]{appendix}
%\usepackage[toc,page]{appendix}
% \usepackage{amsmath,amssymb,amsthm,amsrefs}
% \usepackage{mathtools} % TODO find better for \coloneqq (':=')
% \usepackage{mathpazo}
% \usepackage{inconsolata}
% %\usepackage{euler}
% \usepackage{epigraph} % TODO how does this work
% \usepackage{showkeys} % TODO http://texdoc.net/texmf-dist/doc/latex/tools/showkeys.pdf
% \usepackage{etoolbox}
% \usepackage{ifthen} % one route to argument overloading
%
% %%%%%%%%%%%%%%%%%%
% %% Nomenclature %%
% %%
\usepackage[intoc]{nomencl}
% \usepackage{nomencl}
%\makenomenclature
%
% %%%%%%%%%%%%%%
% %% graphics %%
% %%
% %% TODO https://www.sharelatex.com/learn/Inserting_Images
% \usepackage{graphicx}
% \usepackage{float}
% \graphicspath{{graphics/}{../graphics/}} % relative to both / and sections/
%
% %%%%%%%%%%%%%%%%%%%%%%%%%%%%%%%%%%%
% %% TODO learn correct usage of tikz
% %% tikz
% \usepackage{tikz-cd}
% %% Functions
% %% \begin{tikzcd}[column sep= small,row sep=0ex]
% %%     M_f \colon \pi_1(S \smallsetminus \{y_1, \dots, y_n\}, y) \arrow[r]& \Bij(f^{-1}(y)) \\
% %%    \gamma \arrow[r, mapsto]                                   & M_f(\gamma) = \sigma_{\gamma}^{-1}
% %% \end{tikzcd}
% %%
% %% \begin{align*}
% %%   M_f \colon \pi_1(S \smallsetminus \{y_1, \dots, y_n\}, y) & \longrightarrow \Bij(f^{-1}(y)) \\
% %%   \gamma & \longmapsto M_f(\gamma) = \sigma_\gamma^{-1}
% %% \end{align*}
% %%
% %% Or this:\medskip
% %%
% %% Let $ S' =S \smallsetminus \{y_1, \dots, y_n\} $. Define
% %% $ \begin{aligned}[t]
% %% M_f \colon \pi_1(S', y) &\longrightarrow \Bij(f^{-1}(y)) \\
% %% \gamma &\longmapsto M_f(\gamma) = \sigma_\gamma^{-1}
% %% \end{aligned} $
%
%
% %%%%%%%%%%%%%%%%%%%%%%%%%%%%%%%%%%%%%%%%%%%%%%%%%%
% %% TODO find a better way to handle subsections %%
% %% - glg.tex file inclusions
% %% - introduce nomenclature db using section tags + groups
% %% - standardize labels throughout project
% %% - place the above into an array and autogen glg.tex
% %% - glg.tex file inclusions
% %% - introduce nomenclature db using section tags + groups
% %% - standardize labels throughout project
%
% % \usepackage{etoolbox}
% %  \renewcommand\nomgroup[1]{%
% %    \item[\bfseries
% %      \ifstrequal{#1}{\leca}{\lecatitle}{%
% %      \ifstrequal{#1}{\lecb}{\lecbtitle}{%
% %      \ifstrequal{#1}{\lecc}{\lecctitle}{%
% %      }}}%
% %   ]}
% %%%%%%%%%%%%%%%%%%%%%%%
% %% Table of contents %%
% %%
% %% TODO
% %% - fix nomenclature
% %%
% %% \makeatletter
% %% \def\thenomenclature{%
% %%   \section*{\nomname}
% %%   \if@intoc\addcontentsline{toc}{section}{\nomname}\fi%
% %% \nompreamble
% %% \list{}{%
% %% \labelwidth\nom@tempdim
% %% \leftmargin\labelwidth
% %% \advance\leftmargin\labelsep
% %% \itemsep\nomitemsep
% %% \let\makelabel\nomlabel}}
% %% \makeatother
%
%
% %%%%%%%%%%%%%%%%%%%%%%%%%%
% %% Formatting / Display %%
% %%
% % \newcommand{\HRule}{\rule{\linewidth}{0.5mm}}
% \numberwithin{equation}{section}
% % TODO understand what this does
% % (something like number equations within sections)
%
% %%%%%%%%%%%%%
% %% urls/links
% % \usepackage{hyperref}
%
% % examples
% % c.f. \hyperref[mainlemma1]{lemma \ref*{mainlemma} }.
% % take a look at my website \url{http://raeez.com}
% % it never hurts to \href{http://wiki.org/RTFM}{RTFM}
% % I can be reached at
% % \href{mailto:this_is_a_false_addr@raeez.com}{this\_is\_a\_false\_addr@raeez.com}
%
% %%%%%%%%%%%%%%%%%%%%%%%%%%
% %% Theorem Environments %%
% %%
% \newtheorem{thm}{Theorem}[section]
% %\newtheorem{prop}[thm]{Proposition}
% %\newtheorem{lem}[thm]{Lemma}
% \newtheorem{cor}[thm]{Corollary}
% \theoremstyle{remark}
% \newtheorem{rmk}[thm]{Remark}
% \theoremstyle{definition}
% \newtheorem{defn}[thm]{Definition}
% \newtheorem{ex}[thm]{Example}
% \newtheorem{exc}[thm]{Exercise}
% \newtheorem{conj}[thm]{Conjecture}
%
% %%%%%%%%%%%%%%%%%%
% %% Nomenclature %%
% %%
% %% TODO figure out better solution
% %% https://tex.stackexchange.com/questions/361373/nomenclature-entry-in-toc-not-indented-like-a-chapter/361376
%
% %%%%%%%%%%%%%%%%%%%%%%%%%%%%%%%%%%%%
% %% Modify nomenclature generation %%
% %% 1. SI Units
% %% 2. titled groups
% %% 3. enforce manual order
%
% %% 1. Enable SI units
% %% \usepackage{siunitx}
% %% \sisetup{
% %% inter-unit-product=\ensuremath{{}\cdot{}},
% %% per-mode=symbol
% %% }
% %% \nc{\nomunit}[1]{\renewcommand{\nomentryend}{\hspace*{\fill}#1}}
%
% %% c.f. https://tex.stackexchange.com/questions/118114/commands-that-may-take-a-variable-number-of-arguments
%
%
% %% 2. titled groups
%
% %% TODO 3. manual ordering
%
% %% TODO why is % often used before a newline?
%
% %% TIP wrap long descriptions in a \parbox e.g.
% %% \nm[x]{$x$}{\parbox[t]{.75\textwidth}{Unknown variable with a very very
% %% very very very very very very long description}\nomunit{\si{\second}}}
%
% %% The following implements grouping in the nomenclature preamble
% %% c.f. 1. https://tex.stackexchange.com/questions/166556/how-to-make-a-clean-and-grouped-nomenclature-list
% %%      2. https://tex.stackexchange.com/questions/310128/grouped-nomenclature
% %%      3. https://tex.stackexchange.com/questions/318850/grouping-nomenclature-elements
% %%      4. https://www.sharelatex.com/learn/Nomenclatures
%
% %% \begin{document}
% %% \mbox{}
% %%
% %% \nm[A, 02]{$c$}{Speed of light in a vacuum inertial system}
% %% \nm[A, 03]{$h$}{Plank Constant}
% %% \nm[A, 01]{$g$}{Gravitational Constant}
% %% \nm[B, 03]{$\mathbb{R}$}{Real Numbers}
% %% \nm[B, 02]{$\mathbb{C}$}{Complex Numbers}
% %% \nm[B, 01]{$\mathbb{H}$}{Octonions}
% %% \nm[C]{$V$}{Constant Volume}
% %% \nm[C]{$\rho$}{Friction Index}
\usepackage[all,cmtip]{xy}
\usepackage{lscape}
\usepackage{amsfonts}
\usepackage{amssymb}
\usepackage{xcolor}
\usepackage{framed}
\usepackage{amsmath}
\usepackage{comment}
\usepackage{amsthm}
\usepackage{pdflscape}
\usepackage{hyperref}
%TCIDATA{OutputFilter=latex2.dll}
%TCIDATA{Version=5.50.0.2960}
%TCIDATA{LastRevised=Thursday, June 09, 2016 23:26:41}
%TCIDATA{SuppressPackageManagement}
%TCIDATA{<META NAME="GraphicsSave" CONTENT="32">}
%TCIDATA{<META NAME="SaveForMode" CONTENT="1">}
%TCIDATA{BibliographyScheme=Manual}
%BeginMSIPreambleData
\providecommand{\U}[1]{\protect\rule{.1in}{.1in}}
%EndMSIPreambleData
\newcommand{\Ker}{\operatorname*{Ker}}
\newcommand{\id}{\operatorname*{id}}
\newcommand{\inc}{\operatorname*{inc}}
\newcommand{\gr}{\operatorname*{gr}}
\newcommand{\Hom}{\operatorname*{Hom}}
\newcommand{\calA}{\mathcal A}
\newcommand{\arinj}{\ar@{_{(}->}}
\newcommand{\arsurj}{\ar@{->>}}
\newcommand{\arelem}{\ar@{|->}}
\newcommand{\fraka}{\mathfrak{a}}
\newcommand{\frakb}{\mathfrak{b}}
\newcommand{\frakc}{\mathfrak{c}}
\newcommand{\PBW}{\operatorname*{PBW}}
\newcommand{\xycs}{\xymatrixcolsep}
\newcommand{\xyrs}{\xymatrixrowsep}
\theoremstyle{definition}
\newtheorem{theo}{Theorem}[subsection]
\newenvironment{theorem}[1][]
{\begin{theo}[#1]\begin{leftbar}}
{\end{leftbar}\end{theo}}
\newtheorem{impnt}[theo]{Important Notice}
\newenvironment{impnot}[1][]
{\begin{impnt}[#1]\begin{leftbar}\color{blue}}
{\color{black}\end{leftbar}\end{impnt}}
\newtheorem{lem}[theo]{Lemma}
\newenvironment{lemma}[1][]
{\begin{lem}[#1]\begin{leftbar}}
{\end{leftbar}\end{lem}}
\newtheorem{prop}[theo]{Proposition}
\newenvironment{proposition}[1][]
{\begin{prop}[#1]\begin{leftbar}}
{\end{leftbar}\end{prop}}
\newtheorem{exe}[theo]{Exercise}
\newenvironment{exercise}[1][]
{\begin{exe}[#1]\begin{leftbar}}
{\end{leftbar}\end{exe}}
\newtheorem{defi}[theo]{Definition}
\newenvironment{definition}[1][]
{\begin{defi}[#1]\begin{leftbar}}
{\end{leftbar}\end{defi}}
\newtheorem{remk}[theo]{Remark}
\newenvironment{remark}[1][]
{\begin{remk}[#1]\begin{leftbar}}
{\end{leftbar}\end{remk}}
\newtheorem{coro}[theo]{Corollary}
\newenvironment{corollary}[1][]
{\begin{coro}[#1]\begin{leftbar}}
{\end{leftbar}\end{coro}}
\newtheorem{conv}[theo]{Convention}
\newenvironment{Convention}[1][]
{\begin{conv}[#1]\begin{leftbar}}
{\end{leftbar}\end{conv}}
\newtheorem{warn}[theo]{Warning}
\newenvironment{Warning}[1][]
{\begin{warn}[#1]\begin{leftbar}}
{\end{leftbar}\end{warn}}
\newtheorem{example}[theo]{Example}
\voffset=-0.5cm
\hoffset=-0.7cm
\newenvironment{verlong}{}{}
\newenvironment{vershort}{}{}
\newenvironment{noncompile}{}{}
\includecomment{verlong}
\excludecomment{vershort}
\excludecomment{noncompile}
\setlength\textheight{24cm}
\setlength\textwidth{15.5cm}
