% -------------------------------------------------------------
% NOTE ON THE DETAILED AND SHORT VERSIONS:
% -------------------------------------------------------------
% This paper comes in two versions, a detailed and a short one.
% The short version should be more than sufficient for any
% reasonable use; the detailed one was written purely to
% convince the author of its correctness.
% To switch between the two versions, find the line containing
% "\newenvironment{noncompile}{}{}" in this LaTeX file.
% Look at the two lines right beneath this line.
% To compile the detailed version, they should be as follows:
%   \includecomment{verlong}
%   \excludecomment{vershort}
% To compile the short version, they should be as follows:
%   \excludecomment{verlong}
%   \includecomment{vershort}
% As a rule, the line
%   \excludecomment{noncompile}
% should stay as it is.
% -------------------------------------------------------------
% NOTES ON SOME HACKS USED IN THIS FILE:
% -------------------------------------------------------------
% One of my pet peeves with amsthm is its use of italics in the theorem and
% proposition environments; this makes math and text indistinguishable in said
% enviroments. To avoid this, I redefine the enviroments to use the standard
% font and to use a hanging indent, along with a bold vertical bar to its
% left, to distinguish these environments from surrounding text. (Along with
% the advantage of distinguishing math from text, this also allows nesting
% several such environments inside each other, like a definition inside a
% remark. I'm not sure how good of an idea this is, though. There are also
% downsides related to the hanging indentation, such as footnotes out of it
% being painful to do right.) This is done starting from the line
%   \theoremstyle{definition}
% and until the line
%   {\end{leftbar}\end{exmp}}

\documentclass
[numbers=enddot,12pt,final,onecolumn,notitlepage]{amsbook}%

%%%%%%%%%%%%%%%%%%%%%%%%
%% project modularity %%
%%
\usepackage[subpreambles=true]{standalone}
\usepackage{import} % more elegant than \input for standalone documents

%%%%%%%%%%%%%%%%%%%%%%%%%%%%%%%%%%
%% design, formatting and fonts %%
%%

%\usepackage[titletoc]{appendix}
%\usepackage[toc,page]{appendix}
% \usepackage{amsmath,amssymb,amsthm,amsrefs}
% \usepackage{mathtools} % TODO find better for \coloneqq (':=')
% \usepackage{mathpazo}
% \usepackage{inconsolata}
% %\usepackage{euler}
% \usepackage{epigraph} % TODO how does this work
% \usepackage{showkeys} % TODO http://texdoc.net/texmf-dist/doc/latex/tools/showkeys.pdf
% \usepackage{etoolbox}
% \usepackage{ifthen} % one route to argument overloading
%
% %%%%%%%%%%%%%%%%%%
% %% Nomenclature %%
% %%
\usepackage[intoc]{nomencl}
% \usepackage{nomencl}
%\makenomenclature
%
% %%%%%%%%%%%%%%
% %% graphics %%
% %%
% %% TODO https://www.sharelatex.com/learn/Inserting_Images
% \usepackage{graphicx}
% \usepackage{float}
% \graphicspath{{graphics/}{../graphics/}} % relative to both / and sections/
%
% %%%%%%%%%%%%%%%%%%%%%%%%%%%%%%%%%%%
% %% TODO learn correct usage of tikz
% %% tikz
% \usepackage{tikz-cd}
% %% Functions
% %% \begin{tikzcd}[column sep= small,row sep=0ex]
% %%     M_f \colon \pi_1(S \smallsetminus \{y_1, \dots, y_n\}, y) \arrow[r]& \Bij(f^{-1}(y)) \\
% %%    \gamma \arrow[r, mapsto]                                   & M_f(\gamma) = \sigma_{\gamma}^{-1}
% %% \end{tikzcd}
% %%
% %% \begin{align*}
% %%   M_f \colon \pi_1(S \smallsetminus \{y_1, \dots, y_n\}, y) & \longrightarrow \Bij(f^{-1}(y)) \\
% %%   \gamma & \longmapsto M_f(\gamma) = \sigma_\gamma^{-1}
% %% \end{align*}
% %%
% %% Or this:\medskip
% %%
% %% Let $ S' =S \smallsetminus \{y_1, \dots, y_n\} $. Define
% %% $ \begin{aligned}[t]
% %% M_f \colon \pi_1(S', y) &\longrightarrow \Bij(f^{-1}(y)) \\
% %% \gamma &\longmapsto M_f(\gamma) = \sigma_\gamma^{-1}
% %% \end{aligned} $
%
%
% %%%%%%%%%%%%%%%%%%%%%%%%%%%%%%%%%%%%%%%%%%%%%%%%%%
% %% TODO find a better way to handle subsections %%
% %% - glg.tex file inclusions
% %% - introduce nomenclature db using section tags + groups
% %% - standardize labels throughout project
% %% - place the above into an array and autogen glg.tex
% %% - glg.tex file inclusions
% %% - introduce nomenclature db using section tags + groups
% %% - standardize labels throughout project
%
% % \usepackage{etoolbox}
% %  \renewcommand\nomgroup[1]{%
% %    \item[\bfseries
% %      \ifstrequal{#1}{\leca}{\lecatitle}{%
% %      \ifstrequal{#1}{\lecb}{\lecbtitle}{%
% %      \ifstrequal{#1}{\lecc}{\lecctitle}{%
% %      }}}%
% %   ]}
% %%%%%%%%%%%%%%%%%%%%%%%
% %% Table of contents %%
% %%
% %% TODO
% %% - fix nomenclature
% %%
% %% \makeatletter
% %% \def\thenomenclature{%
% %%   \section*{\nomname}
% %%   \if@intoc\addcontentsline{toc}{section}{\nomname}\fi%
% %% \nompreamble
% %% \list{}{%
% %% \labelwidth\nom@tempdim
% %% \leftmargin\labelwidth
% %% \advance\leftmargin\labelsep
% %% \itemsep\nomitemsep
% %% \let\makelabel\nomlabel}}
% %% \makeatother
%
%
% %%%%%%%%%%%%%%%%%%%%%%%%%%
% %% Formatting / Display %%
% %%
% % \newcommand{\HRule}{\rule{\linewidth}{0.5mm}}
% \numberwithin{equation}{section}
% % TODO understand what this does
% % (something like number equations within sections)
%
% %%%%%%%%%%%%%
% %% urls/links
% % \usepackage{hyperref}
%
% % examples
% % c.f. \hyperref[mainlemma1]{lemma \ref*{mainlemma} }.
% % take a look at my website \url{http://raeez.com}
% % it never hurts to \href{http://wiki.org/RTFM}{RTFM}
% % I can be reached at
% % \href{mailto:this_is_a_false_addr@raeez.com}{this\_is\_a\_false\_addr@raeez.com}
%
% %%%%%%%%%%%%%%%%%%%%%%%%%%
% %% Theorem Environments %%
% %%
% \newtheorem{thm}{Theorem}[section]
% %\newtheorem{prop}[thm]{Proposition}
% %\newtheorem{lem}[thm]{Lemma}
% \newtheorem{cor}[thm]{Corollary}
% \theoremstyle{remark}
% \newtheorem{rmk}[thm]{Remark}
% \theoremstyle{definition}
% \newtheorem{defn}[thm]{Definition}
% \newtheorem{ex}[thm]{Example}
% \newtheorem{exc}[thm]{Exercise}
% \newtheorem{conj}[thm]{Conjecture}
%
% %%%%%%%%%%%%%%%%%%
% %% Nomenclature %%
% %%
% %% TODO figure out better solution
% %% https://tex.stackexchange.com/questions/361373/nomenclature-entry-in-toc-not-indented-like-a-chapter/361376
%
% %%%%%%%%%%%%%%%%%%%%%%%%%%%%%%%%%%%%
% %% Modify nomenclature generation %%
% %% 1. SI Units
% %% 2. titled groups
% %% 3. enforce manual order
%
% %% 1. Enable SI units
% %% \usepackage{siunitx}
% %% \sisetup{
% %% inter-unit-product=\ensuremath{{}\cdot{}},
% %% per-mode=symbol
% %% }
% %% \nc{\nomunit}[1]{\renewcommand{\nomentryend}{\hspace*{\fill}#1}}
%
% %% c.f. https://tex.stackexchange.com/questions/118114/commands-that-may-take-a-variable-number-of-arguments
%
%
% %% 2. titled groups
%
% %% TODO 3. manual ordering
%
% %% TODO why is % often used before a newline?
%
% %% TIP wrap long descriptions in a \parbox e.g.
% %% \nm[x]{$x$}{\parbox[t]{.75\textwidth}{Unknown variable with a very very
% %% very very very very very very long description}\nomunit{\si{\second}}}
%
% %% The following implements grouping in the nomenclature preamble
% %% c.f. 1. https://tex.stackexchange.com/questions/166556/how-to-make-a-clean-and-grouped-nomenclature-list
% %%      2. https://tex.stackexchange.com/questions/310128/grouped-nomenclature
% %%      3. https://tex.stackexchange.com/questions/318850/grouping-nomenclature-elements
% %%      4. https://www.sharelatex.com/learn/Nomenclatures
%
% %% \begin{document}
% %% \mbox{}
% %%
% %% \nm[A, 02]{$c$}{Speed of light in a vacuum inertial system}
% %% \nm[A, 03]{$h$}{Plank Constant}
% %% \nm[A, 01]{$g$}{Gravitational Constant}
% %% \nm[B, 03]{$\mathbb{R}$}{Real Numbers}
% %% \nm[B, 02]{$\mathbb{C}$}{Complex Numbers}
% %% \nm[B, 01]{$\mathbb{H}$}{Octonions}
% %% \nm[C]{$V$}{Constant Volume}
% %% \nm[C]{$\rho$}{Friction Index}
\usepackage[all,cmtip]{xy}
\usepackage{lscape}
\usepackage{amsfonts}
\usepackage{amssymb}
\usepackage{xcolor}
\usepackage{framed}
\usepackage{amsmath}
\usepackage{comment}
\usepackage{amsthm}
\usepackage{pdflscape}
\usepackage{hyperref}
%TCIDATA{OutputFilter=latex2.dll}
%TCIDATA{Version=5.50.0.2960}
%TCIDATA{LastRevised=Thursday, June 09, 2016 23:26:41}
%TCIDATA{SuppressPackageManagement}
%TCIDATA{<META NAME="GraphicsSave" CONTENT="32">}
%TCIDATA{<META NAME="SaveForMode" CONTENT="1">}
%TCIDATA{BibliographyScheme=Manual}
%BeginMSIPreambleData
\providecommand{\U}[1]{\protect\rule{.1in}{.1in}}
%EndMSIPreambleData
\newcommand{\Ker}{\operatorname*{Ker}}
\newcommand{\id}{\operatorname*{id}}
\newcommand{\inc}{\operatorname*{inc}}
\newcommand{\gr}{\operatorname*{gr}}
\newcommand{\Hom}{\operatorname*{Hom}}
\newcommand{\calA}{\mathcal A}
\newcommand{\arinj}{\ar@{_{(}->}}
\newcommand{\arsurj}{\ar@{->>}}
\newcommand{\arelem}{\ar@{|->}}
\newcommand{\fraka}{\mathfrak{a}}
\newcommand{\frakb}{\mathfrak{b}}
\newcommand{\frakc}{\mathfrak{c}}
\newcommand{\PBW}{\operatorname*{PBW}}
\newcommand{\xycs}{\xymatrixcolsep}
\newcommand{\xyrs}{\xymatrixrowsep}
\theoremstyle{definition}
\newtheorem{theo}{Theorem}[subsection]
\newenvironment{theorem}[1][]
{\begin{theo}[#1]\begin{leftbar}}
{\end{leftbar}\end{theo}}
\newtheorem{impnt}[theo]{Important Notice}
\newenvironment{impnot}[1][]
{\begin{impnt}[#1]\begin{leftbar}\color{blue}}
{\color{black}\end{leftbar}\end{impnt}}
\newtheorem{lem}[theo]{Lemma}
\newenvironment{lemma}[1][]
{\begin{lem}[#1]\begin{leftbar}}
{\end{leftbar}\end{lem}}
\newtheorem{prop}[theo]{Proposition}
\newenvironment{proposition}[1][]
{\begin{prop}[#1]\begin{leftbar}}
{\end{leftbar}\end{prop}}
\newtheorem{exe}[theo]{Exercise}
\newenvironment{exercise}[1][]
{\begin{exe}[#1]\begin{leftbar}}
{\end{leftbar}\end{exe}}
\newtheorem{defi}[theo]{Definition}
\newenvironment{definition}[1][]
{\begin{defi}[#1]\begin{leftbar}}
{\end{leftbar}\end{defi}}
\newtheorem{remk}[theo]{Remark}
\newenvironment{remark}[1][]
{\begin{remk}[#1]\begin{leftbar}}
{\end{leftbar}\end{remk}}
\newtheorem{coro}[theo]{Corollary}
\newenvironment{corollary}[1][]
{\begin{coro}[#1]\begin{leftbar}}
{\end{leftbar}\end{coro}}
\newtheorem{conv}[theo]{Convention}
\newenvironment{Convention}[1][]
{\begin{conv}[#1]\begin{leftbar}}
{\end{leftbar}\end{conv}}
\newtheorem{warn}[theo]{Warning}
\newenvironment{Warning}[1][]
{\begin{warn}[#1]\begin{leftbar}}
{\end{leftbar}\end{warn}}
\newtheorem{example}[theo]{Example}
\voffset=-0.5cm
\hoffset=-0.7cm
\newenvironment{verlong}{}{}
\newenvironment{vershort}{}{}
\newenvironment{noncompile}{}{}
\includecomment{verlong}
\excludecomment{vershort}
\excludecomment{noncompile}
\setlength\textheight{24cm}
\setlength\textwidth{15.5cm}

%%%%%%%%%%%%%%%%%%%%%%%%
%% Markup Readability %%
%%
%% macros
\newcommand\nc{\newcommand}
\newcommand\rc{\renewcommand}

%% nomenclature
%\newcommand\nm{\nomenclature}
\newcommand\nm{} % temporarily disable standard \nomenclature
%\newcommand\mathnm[2]{\nm[#1]{}}

%% shorten the macro syntax
\newcommand\mc[1]{\mathcal{#1}}
\newcommand\mbb[1]{\mathbb{#1}}
\newcommand\mbf[1]{\mathbf{#1}}
\newcommand\mrm[1]{\mathrm{#1}}

\newcommand\image[1]{
  \begin{figure}[H]\label{fig-#1}
    \centering
    \includegraphics[scale=.5]{#1}
  \end{figure}
}
\newcommand\imageopt[3]{ % \imageopt #1-filename #2-scale #3-caption
  \begin{figure}[H]
    \centering
    \ifstrequal{#2}{}{%if no scale provided
      \includegraphics{#1}}{ % don't set the scale
      }{\includegraphics[scale=#2]{#1}}\label{fig:#1}
    \ifstrequal{#3}{}{%if no caption provided
      }{ % don't try to set one
      }{\caption{#3}} % otherwise do
  \end{figure}}

\newcommand\mathsc[1]{\text{\normalfont\scshape#1}}

%% common symbols
\newcommand\lb{\left[}  \newcommand\rb{\right]}  % [ ]
\newcommand\llb{\lb\lb}  \newcommand\rrb{\rb\rb} % [[ ]]

\newcommand\lp{\left(}  \newcommand\rp{\right)} % ( )
\newcommand\llp{\lp\lp} \newcommand\rrp{\rp\rp} % (( ))

\newcommand\twotuple[2]{\lp #1,#2\rp}   % TODO implement variable number of args
\newcommand\setpresent[2]{\{ #1 | #2\}} % TODO typeset correctly

%% Index into Symbols and Notation
%% We collect all notation utilized throughouth this lecture series
%% we the hope their collation facillitates future functional/aesthetic changes

%%%%%%%%%%%%%%%%%%%%%%%%%%%%%%%%%%%%
%% TODO
%% - clarify class formation group k
%%    * little/big, more readable
%%    * visually distinguish notation for C_L / element
%% - Generate a glossary
%% - mark up structure for nomenclature / glossary inline
%%

\nc\Cat[1]{\mathbf{#1}} % typeset name of category
% TODO fit these in
\nc\G{G}           % reductive group
% algebraic fields, formal power series and their fraction fields
\nc\ringint{\mathcal{O}}           % abstract ring of integers
\nc\grk{k}                         %abstract ground field
\nc\N{\mathbb{N}}
\nc\Z{\mathbb{Z}}
\nc\Q{\mathbb{Q}}
\nc\R{\mathbb{R}}
\nc\C{\mathbb{C}}
\nc{\FnS}{\grk\lp\Sigma\rp}
\nc\laurentpoly{\lb x,x^{-1}\rb}
\nc\laurentser{\llp~z\rrp}
\nc\fpowerser{\llb~z\rrb}
\nc\klocint{\grk\fpowerser}
\nc\Klocfrac{K\laurentser}
\nc\tensor{\otimes} % algebraic tensor product

%\nc\Spec[1]{\mathrm{Spec} (#1)}
%\nc\Specf[1]{\mathrm{Specf} (#1)}

% TODO figure out optimal manner of handling infix macros
% something like \infixnewcommand{\T}{{bunewcommandhofstuff} {#1} {bunewcommandhofstuff} {#2} {bunewcommandhofstuff}}

% TODO read more about macros
% https://en.wikibooks.org/wiki/LaTeX/Macros

% relations, operations, unitary and binary symbols etc.
\nc\isom{\simeq} % TODO find optimal

\nc\ov[1]{/ {#1}}%'over' as in Scheme over k or galois extension over k

\nc\restr[2]{{% we make the whole thing an ordinary symbol
  \left.\kern-\nulldelimiterspace% automatically resize the bar with \right
  #1% the funewcommandtion
  \vphantom{\big|} % pretend it's a little taller at normal size
  \right|_{#2} % this is the delimiter
  }}

\nc\shf[1]{\mathcal{#1}}
\nc\idshf[1]{\mathcal{#1}}

\nc\projline{\mathbb{P}^1}

\nc\K{\mathbb{K}}

\nc\polyring[2]{R}

\nc\classgrp[1]{C_{#1}}
\nc\F{\mathbb{F}}
\nc\idealgenby[1]{\langle#1\rangle}
\nc\D{\mathbf{d}}
\nc\pt{\mathbf{pt}}
\nc\Gr{\textrm{Gr}} % affine grassmannian.

  \nc\Ga{\mathbb{G}_a}

  \nc\Gm{\mathbb{G}_m}

\nc\lpgrass{\mathcal{G}}
\nc\glpgrass{\mathcal{G}^P (I,Q)}

%, commonly denoted $\Gr_a$ or $G_{ K / {\ringint}$.}
% TODO = \mathbb{G}_{\mathcal{G}} = \mathcal{G}_{\mathcal{K}} \textrm{
% mod } \mathcal{G}_{\mathcal{O}}
% TODO structure nomenclature pre-amble by
% TODO 1. generality / importanewcommande
% TODO 2. chapter/lecture

% TODO group + order intro

%%%%%%%%%%%%%%%%%%%%%
%% classformations %%
%%
\nc\Flds{\Cat{Fields}}
\nc\cartdual{\mc{D}}

\nc\Mat[1]{\mathbf{Mat}_{#1\times#1}}
\nc\RH{\mathbb{\R}\textrm{H}}
\nc\tateshiftedby[1]{\left[#1\right]}
\nc\TateCat{\Cat{J_B}}
\nc\TateCatG{\Cat{J_G}}
\nc\B[1]{\mathbb{B} (#1)}
\nc\ptmod[1]{\mathbf{pt}\ov{#1}}
\nc\Ind{\mathrm{Ind}}
%\rc\Hom{\mathrm{Hom}}
\nc\Map{\mathrm{Map}}
\nc\coH{\mathbf{H}}
\nc\hcoH[3]{\widehat{\coH^{#1}#2,#3}}
\nc\Ab{\Cat{Ab}} % TODO fix
%\nc\Ker{\mathbf{Ker}}
\nc\Coker{\mathbf{Coker}}
\nc\dg{\textrm{DG}}
\rc\Vec{\Cat{Vec}}
\nc\VecF{\Cat{Vec}_{\grk}}
\nc\Coh{\Cat{Coh}}

\nc\Coind{\mathbf{Coind}}
\rc\Ind{\mathbf{Ind}}
\nc\Res{\mathbf{Res}}

%%%%%%%%%%%%%%%%%%%%%
%% classformations %%
%%
\nc\Pic{\mathbf{Pic}}
\nc\AJ{\mathbf{AJ}}
\nc\divisor[1]{\mathbf{#1}}
\nc\divisoridshf[1]{\idshf{I_{\divisor{#1}}}}
\nc\IndProSch{\Cat{IndProSch}}
\nc\FinIndProSch{\Cat{FinIndProSch}}
\nc\Pro{\mathbf{Pro}}
\nc\IndPro{\Ind-\Pro}
\nc\FinIndSch{\Cat{FinIndProSch}}
\nc\A{\Cat{CommIndSch_{\grk}}}

\nc\Fun[2]{#1 (#2)}
\nc\Frac[1]{\mathbf{Frac} (#1)}
\nc\curriedleftarg{\ldots}
\nc\Zp{\Z_p}

\nc\Qp{\mathbb{Q}_p}

% TODO group + order lec 3
% TODO review cardinality issues / small categories etc.
\nc\Set{\Cat{Set}} % Category of Sets

\nc\Hilb{\mathcal{H}\mathbb{ilb}}

\nc\disj{\sqcup} % TODO distinguish bigsqcup and sqcup
\nc\bij{\leftrightarrow}
\nc\sur{\twoheadedrightarrow}
\nc\inj{\rightarrowtail}
\nc\injects{\inj}
% \nc\xinj{\xrightarrowtail} % see sty-glg/commands package
% \nc\xhkinj{\xrightarrowtail} % see sty-glg/commands package
% \begin{document}
%   \[ f : G \xrightarrowtail[{\star}]{\text{\textbf{Grp}}} H \]
% \end{document}

% \xhookrightarrow from mathtools
% \[
% A\xhookrightarrow{} B\qquad A\xhookrightarrow{f\cirenewcommand g} B\qquad
% A \xhookrightarrow[(f\cirenewcommand g)\cirenewcommand h]{} B
% \]
% \xrightarrow from amsmath
% \[
% A\rightarrow{} B\qquad A\xrightarrow{f\cirenewcommand g} B\qquad
% A \xrightarrow[(f\cirenewcommand g)\cirenewcommand h]{} B
% \]

% TODO group, order + split off cohomology.tex

\nc\fst{\ensuremath{2^{\text{\tiny{st}}}}}
\nc\snd{\ensuremath{3^{\text{\tiny{nd}}}}}
\nc\thrd{\ensuremath{3^{\text{\tiny{rd}}}}}
\nc\nth{\ensuremath{n^{\text{\tiny{th}}}}}
% TODO find elegant resolution; possibilities:
% * \(n\)th
% * \usepackage{nth} NB: \nth{3} is different from $\nth{3}$
% * $n$-th
% * $n^{\text{\tiny th}}$
\nc\shfcoH[3]{\mathbf{H}^{#3} (#1,#2)}
\nc\hshfcoH[3]{\widehat{\shfcoH{#1,#2,#3}}}
\nc\Obj{\mathbf{Obj}}
% TODO nomencl Obj
\nc\grpring[2]{#1 \left[ #2 \right]}

\nc\Ext{\mathbf{Ext}}
\nc\Exts[3]{\Ext_{#1}^*\left(#2,#3 \right)}
% TODO nomencl Exts
\nc\dualmod[1]{#1^{*}}
% TODO investigate prefix macros e.g. {A}\Mod
\nc\LModCat[1]{#1-\Cat{LMod}}
\nc\RModCat[1]{#1-\Cat{RMod}}
\nc\indlim{\mathrm{lim}}

% TODO group + order indobjs.tex

\nc\Schemes{\Cat{Schemes}}

\nc\NoethSch{\Cat{NoethSch}}
\nc\NoethIntSch{\Cat{NoethIntSch}}

\nc\Sch[1]{\Cat{Sch}_{#1}}
\nc\IndSch[1]{\Cat{IndSch}_{#1}}

\nc\ModCat[1]{#1-\Cat{Mod}}

%\rc\dim[1]{\ensuremath{\mathbf{dim} (#1)}}

%%%%%%%%%%%%%%%%%%%%%%%%%%
%%
%%
% TODO learn how to typeset arrows in categories
% functors / presheaves / sheaves / group schemes etc.

\nc\grkAlg{\Cat{\grk-Algebras}}
\nc\ZAlg{\Cat{\Z-Algebras}}
%\nc\SL[1]{\mathbf{SL}_{#1}}

%{\ifstrequal{#1}{}{%if no scale provided
%  \SpecialLinearAbr}{


%% Index into Symbols and Notation
%% We collect all notation utilized throughouth this lecture series
%% we the hope their collation facillitates future functional/aesthetic changes

%%%%%%%%%%%%%%%%%%%%%%%%%%%%%%%%%%%%
%% TODO
%% - clarify class formation group k
%%    * little/big, more readable
%%    * visually distinguish notation for C_L / element
%% - Generate a glossary
%% - mark up structure for nomenclature / glossary inline
%%

% TODO fit these in
\nc\cc
{\ensuremath{\mathbf{CC}}} % TODO fix this
\nc\Bun{\ensuremath{\mathrm{Bun}}}

\nc\Spec[1]{\mathrm{Spec} (#1)}
\nc\Specf[1]{\mathrm{Specf} (#1)}

\nc\mK{\mathbf{K}} % TODO abstract out to notation.tex
\nc\Bil{\mathbf{Bil}} % TODO abstract to notation.tex
\nc\IndNSt{\Cat{Indn-Stack}}
% TODO figure out optimal manner of handling infix macros
% something like \infixnewcommand{\T}{{bunewcommandhofstuff} {#1} {bunewcommandhofstuff} {#2} {bunewcommandhofstuff}}

% TODO read more about macros
% https://en.wikibooks.org/wiki/LaTeX/Macros

% relations, operations, unitary and binary symbols etc.

%%%%%%%%%%%%%%%%%%%%%
%% classformations %%
%%
\rc\Pic{\mathbf{Pic}}
\rc\AJ{\mathbf{AJ}}
\rc\divisor[1]{\mathbf{#1}}
\rc\divisoridshf[1]{\idshf{I_{\divisor{#1}}}}
\rc\IndProSch{\Cat{IndProSch}}
\rc\FinIndProSch{\Cat{FinIndProSch}}
\rc\Pro{\mathbf{Pro}}
\rc\IndPro{\Ind-\Pro}
\rc\FinIndSch{\Cat{FinIndProSch}}
\rc\A{\Cat{CommIndSch_{\grk}}}

\rc\Fun[2]{#1( #2 )}
\rc\Frac[1]{\mathbf{Frac}( #1 )}
\rc\curriedleftarg{\ldots}
\rc\Zp{\Z_p}

%also called the formal completion at the point $p$ in the curve $\Spec{\Z}$}
\rc\Qp{\mathbb{Q}_p}

\rc\Hilb{\mathcal{H}\mathbb{ilb}}

% \rc\bij{\leftrightarrow}
% \rc\sur{\twoheadedrightarrow}
\rc\inj{\rightarrowtail}
\rc\injects{\inj}
% \rc\xinj{\xrightarrowtail} % see sty-glg/commands package
% \rc\xhkinj{\xrightarrowtail} % see sty-glg/commands package
% \begin{document}
%   \[ f : G \xrightarrowtail[{\star}]{\text{\textbf{Grp}}} H \]
% \end{document}

% \xhookrightarrow from mathtools
% \[
% A\xhookrightarrow{} B\qquad A\xhookrightarrow{f\cirenewcommand g} B\qquad
% A \xhookrightarrow[(f\cirenewcommand g)\cirenewcommand h]{} B
% \]
% \xrightarrow from amsmath
% \[
% A\rightarrow{} B\qquad A\xrightarrow{f\cirenewcommand g} B\qquad
% A \xrightarrow[(f\cirenewcommand g)\cirenewcommand h]{} B
% \]

% TODO group, order + split off cohomology.tex

\rc\fst{\ensuremath{2^{\text{\tiny{st}}}}}
\rc\snd{\ensuremath{3^{\text{\tiny{nd}}}}}
\rc\thrd{\ensuremath{3^{\text{\tiny{rd}}}}}
\rc\nth{\ensuremath{n^{\text{\tiny{th}}}}}
% TODO find elegant resolution; possibilities:
% * \(n\)th
% * \usepackage{nth} NB: \nth{3} is different from $\nth{3}$
% * $n$-th
% * $n^{\text{\tiny th}}$
\rc \shfcoH[3]{\mathbf{H}^{#3}(#1,#2)}
\rc\hshfcoH[3]{\widehat{\shfcoH{#1,#2,#3}}}
\rc\Obj{\mathbf{Obj}}
% TODO nomencl Obj
\rc\grpring[2]{#1 \left[ #2 \right]}

\rc\Ext{\mathbf{Ext}}
\rc\Exts[3]{\Ext_{#1}^*\left(#2,#3 \right)}
% TODO nomencl Exts
\rc\dualmod[1]{#1^{*}}
% TODO investigate prefix macros e.g. {A}\Mod
\rc\LModCat[1]{#1-\Cat{LMod}}
\rc\RModCat[1]{#1-\Cat{RMod}}
\rc\indlim{\mathrm{lim}}

% TODO group + order indobjs.tex

\rc\Schemes{\Cat{Sch}}

\rc\NoethSch{\Cat{NoethSch}}
\rc\NoethIntSch{\Cat{NoethIntSch}}

\rc\Sch[1]{\Cat{Sch}_{#1}}

\rc\IndSch[1]{\Cat{IndSch}_{#1}}

\rc\ModCat[1]{#1-\Cat{Mod}}
%commutative ring $A$.}

\rc\dim[1]{\ensuremath{\mathbf{dim}( #1 )}}

% TODO learn how to typeset arrows in categories
% functors / presheaves / sheaves / group schemes etc.

\rc\grkAlg{\Cat{\grk-Algebras}}
\rc\ZAlg{\Cat{\Z-Algebras}}
\nc\SL{\mathbf{SL}}
\nc\GL{\mathbf{GL}}
\nc\DMod{\Cat{DMod}}
\nc\IndSt{\Cat{IndStacks}}
\rc\IndNSt{\Cat{Ind}-\textsc{N}-\Cat{Stacks}}

\nc\Curve{\Sigma}
\nc\IxCHilb{\Hilb_{\Curve \times I}}
\nc\HilbCxI{\Hilb_{\Curve \times I}}
\nc\locLb{\mc{L}_{I,Q}}
\nc\Locvbpos{\{V_p\}_{p \in P}}
%\nc\IxCHilb{\Hilb_{\Curve \times I}}
%\nc\locLb{\mc{L}_{I,Q}}

% wip-surfaces.tex
\nc\Tor{\mathbf{Tor}}
%\nc\Ext{\mathbf{Tor}}
\nc\W{\mc{W}}
\nc\T{\mbf{T}}
%\nc\P{\mbb{P}}
\nc\Hirz{\mbf{H}}
%\nc\ringint{\mc{O}}
\nc\Tot{\mbf{Tot}}

%#\rc\P{\ensuremath{\mathbb{P}^2}}
\rc\P{\mathbb{P}}
\nc\Pbd{l_{\infty}}
\nc\blP{\widetilde{\mathbb{P}^2 \ov{\Z_2}}}
\nc\cM{\widetilde{\mathbb{M}(r,c)}}
\nc\M{\mathbb{M}(r,n)}
\rc\T{\mathbf{T}}
\nc\ef{\frac{\varepsilon_1}{\varepsilon_2}}
\nc\Vleft{\V{\sqrt{\frac{\ef}{\ef - 1}}}} \nc\Vright{\V{\sqrt{(1 - \ef)}}}
\nc\Verma[2]{\mathbf{V}_{#1,#2}}
\nc\VermaL[2]{\mathbf{L}_{#1,#2}}
\nc\V[1]{\mathbb{V}_{#1}}

\nc\lie[1]{\mathfrak{#1]}}
\nc\so{\lie{so}}
\nc\su{\lie{su}}
\rc\sl{\lie{sl}}
\nc\gl{\lie{gl}}

% thesis.tex
\rc\P{\mathbb{P}}
\rc\sp{\mathfrak{sp}}
\nc\g{\lie{g}}


\begin{document}

\title{Infinite Dimensional Lie Algebras}
\date{Version 0.47 (\today) (\textsc{not proofread}!)}
\begin{abstract}\textbf{18.747}: \textsc{notes on \textit{infinite-dimensional lie
algebras} held in the spring semesters of 2012 and 2016 at MIT}\end{abstract}

%\documentclass[etingof-lie.tex]{subfiles}
%\begin{document}
\author[1]{Pavel Etingof}
\author[2]{Scribed by Darij Grinberg}
\author[3]{(in the process of transmutation) by Raeez Lorgat}
%\email{raeez@mit.edu}
\urladdr{http://math.raeez.com}
%\end{document}

\maketitle
\tableofcontents
\mbox{}
\nomenclature{$\Klocfrac$}{The field
of laurent series valued in $K$; equivalently, the fraction field of
series valued in $k$; equivalently, the ring of integers of the
completed local field.}
\nomenclature{$\Z$}{The ring of integers.}
\nomenclature{$\Q$}{The rational number field}
\nomenclature{$\R$}{The real number field}
\nomenclature{$\C$}{The complex number field}
\nomenclature{$\Spec{A}, \Specf{A\fpowerser}$}{The respective Prime and formal spectra of the commutative rings $A$ and $A\fpowerser$}
\nomenclature{$I,\idshf{I}$}{For an ideal $I$ in a commutative ring $A$,
$\idshf{I}$ denotes the associated ideal sheaf on $\Spec{A}$}
\nomenclature{$\projline$}{The projective line as algebro-geometric object, for
example, as represented in the functor-of-points yoga by two copies of $\Z[x]$
glued along a common $\Z\laurentpoly$ }
\nomenclature{$\K$}{The abstract total field}
\nomenclature{$\polyring{R}{n}$}{The ring of $R$-valued polynomials in the
formal variables $z_1,\ldots,z_n$ valued in the ring $R$.}
% TODO clarify distinction between polynomials 'in' and 'over' the $z_i$.
% what is the same: choose a convention for the symmetric algebra and its dual
\nomenclature{$ \FnS $}{Function field of a curve $\Sigma$ defined over $\grk$ }
\nomenclature{$\classgrp{\grk}$}{The class formation group associated to $\grk$.}
\nomenclature{$\F_p$}{A finite field of $p$ elements; for example the
quotient ring $\Z \ov{\idealgenby{p}}$}
\nomenclature{$\D = \D_a$}{The formal disc defined over $\grk$ with the
distinguished $\pt = a$ labeling the origin.}
\nomenclature{$\Gr$ }{The affine grassmannian of $G$.}
\nomenclature{$\Ga$}{The additive group as commutative reductive algebraic group}
\nomenclature{$\Gm$}{The multiplicative group as commutative reductive algebraic group.}
\nomenclature{$\lpgrass$}{The classical loop grassmannian of $G$.}
\nomenclature{$\glpgrass$}{The \textit{generalized} loop grassmannian of $G$}
\nomenclature{$\Flds$}{The Category of Fields}
\nomenclature{$\cartdual$}{Cartier duality on a category } % TODO say more}
\nomenclature{$\Mat{n}\grk$}{$n\times n$ matrices with coefficients in $\grk$}
\nomenclature{$\RH^*$}{Right derived functor of homology}
\nomenclature{$X\tateshiftedby{n}$}{The nth Tate Shift}
\nomenclature{$\TateCat$}{The tate category associated to a group $H$, defined as the
categorical quotient $\mathcal{A}^G_{N_G(\mc{A})}$} %N_G(\mathcal{A}
\nomenclature{$\B{G}$}{everyone's favourite quotient category}
\nomenclature{$\ptmod{G}$}{everyone's favourite quotient category}
%\nomenclature{$\hcoH{G,A,n}$}{blah}
\nomenclature{$\Vec_{\grk}$}{Category of Vector Spaces over $\grk$}
\nomenclature{$\divisor{p},\divisoridshf{\divisor{p}}$}{A divisor on some ambient space
along with its ideal sheaf}
\nomenclature{$\AJ$}{The Abel Jacobi map taking a divisor $\divisor{p}$
  supported on a space $X$ to $\AJ: \divisor{p} \mapsto \ringint_{X}
(-\divisor{p})\divisoridshf{I_{\divisor{p}}}$}
\nomenclature{$\IndProSch$}{The category of $\Ind$-$\Pro$-Schemes}
\nomenclature{$\FinIndProSch$}{The category of \textit{finite} $\IndPro$-Schemes}
\nomenclature{$\FinIndSch$}{The category of \textit{finite} $\Ind$-Schemes}
\nomenclature{$\A$}{The category of commutative indschemes over a ring $\grk$}
\nomenclature{$\Zp$}{The inverse limit of the inverse system of rings $\Z \ov{p^n \Z}$,
also called the formal completion at the point $p$ in the curve $\Spec{\Z}$}
\nomenclature{$\Set$}{The (small) category of Sets}
\nomenclature{$\Set$}{The (small) category of Sets}
\nomenclature{$\Hilb_{X}$}{The hilbert scheme of $X$ representing the moduli of finite length subschemes of $X$.}
%\nomenclature{$\shfcoH{X,\shf{I},n}$}{The $n$th cohomology of the sheaf $\shf{I}$ over the space $X$}.
\nomenclature{$\grpring{Z}{G}$}{The group ring of $G$}
\nomenclature{$\Bun G$}{The moduli space of $G$ bundles.}
\nomenclature[coh]{$\dualmod{M}$}{The dual module $\Hom{A}(M,A)$ associated to an object $M \in \ModCat{A}$ for some ring $A$}
\nomenclature{$\LModCat{A}$}{The category of left modules over a ring $A$.}
\nomenclature{$\RModCat{A}$}{The category of right modules over a ring $A$.}
\nomenclature{$\Schemes$}{The category of Schemes}
\nomenclature{$\NoethSch$}{The category of Noetherian Schemes}
\nomenclature{$\NoethIntSch$}{The category of Integral Noetherian Schemes}
\nomenclature{$\Sch{\grk}$}{The category of Schemes defined over $\grk$}
\nomenclature{$\IndSch{\grk}$}{The category of IndSchemes defined over $\grk$}
\nomenclature{$\ModCat{A}$}{The category of modules over a ring $A$.}
\nomenclature{$\SL_2$}{The reductive affine algebraic group scheme
  associated to the Special Linear Group of invertible determinant $1$ matrices
  i.e.\ the fiber $det^{-1}(-1)$ in $\Mat{2}$. Equivalently characterized via a
  \textit{functor of points} formalism e.g.\ $\grkAlg \rightarrow \Set$
represented in $\ZAlg$ by $\grpring{\Z}{\SL_2} \isom$ }
  %\polyring{Z}{4}\ov{\idealgenby{z_1z_3 - z_2z_4 - 1}}$}
\nomenclature{$G$}{A reductive group.}
\nomenclature{$\ringint$}{The local ring $\klocint$ ring of formal power series.}
% TODO frame as localization / indsystem
\nomenclature{$\grk$}{The ground field,
  often the complex numbers $\C$, and
  almost always embedded in the total field
  $K$.}
%\nomenclature[\ageom]{$\Spec{A}, \Specf{A\fpowerser}$}{The respective Prime and formal spectra of the commutative rings $A$ and $A\fpowerser$}
%\nomenclature{$I,\idshf{I}$}{For an ideal $I$ in a commutative ring $A$,
%$\idshf{I}$ denotes the associated ideal sheaf on $\Spec{A}$}
%\nomenclature{$\projline$}{The projective line as algebro-geometric object, for
%example, as represented in the functor-of-points yoga by two copies of $\Z[x]$
%glued along a common $\Z\laurentpoly$ }
%\nomenclature[\seccfm]{$\K$}{The abstract total field}
%\nomenclature[\ageom]{$\polyring{R}{n}$}{The ring of $R$-valued polynomials in the
%formal variables $z_1,\ldots,z_n$ valued in the ring $R$.}
%\nomenclature{$G$}{A reductive group.}
% algebraic fields, formal power series and their fraction fields
%\nomenclature[\ageom]{$\ringint$}{The local ring $\klocint$ ring of formal power
%series valued in $k$; equivalently, the ring of integers of the
%completed local field.}
%\nomenclature[\ageom]{$\grk$}{The ground field,
%  often the complex numbers $\C$, and
%  almost always embedded in the total field
%  $K$.}
%\nomenclature[\ageom]{$\Klocfrac$}{The field
%  of laurent series valued in $K$; equivalently, the fraction field of
%$\ringint$.}
%\nomenclature{$\Z$}{The ring of integers}
%\nomenclature{$\Q$}{The rational number field}
%\nomenclature{$\R$}{The real number field}
%\nomenclature{$\C$}{The complex number field}
%\nomenclature{$ \FnS $}{Function field of a curve $\Sigma$ defined over $\grk$ }
%\nomenclature{$\classgrp{\grk}$}{The class formation group associated to $\grk$.}
%\nomenclature{$\F_p$}{A finite field of $p$ elements; for example the
%\nomenclature{$\D = \D_a$}{The formal disc defined over $\grk$ with the
%distinguished $\pt = a$ labeling the origin.}
%\nomenclature{$\Gr$ }{The affine grassmannian of $G$.}
%\nomenclature{$\Ga$}{The additive group as commutative reductive algebraic group}
%\nomenclature{$\Set$}{The (small) category of Sets}
%\nomenclature{$\Qp$}{The fraction field of the ring of $p$-adic integers, i.e. $\Frac{\Zp}$}
%\nomenclature{$\Zp$}{The inverse limit of the inverse system of rings $\Z \ov{p^n \Z}$,
%\nomenclature[\seccfm]{$\A$}{The category of commutative indschemes over a ring $\grk$}
%\nomenclature[\seccfm]{$\FinIndSch$}{The category of \textit{finite} $\Ind$-Schemes}
%\nomenclature[\seccfm]{$\FinIndProSch$}{The category of \textit{finite} $\IndPro$-Schemes}
%\nomenclature{$\Hilb_{X}$}{The hilbert scheme of $X$ representing the moduli of finite length subschemes of $X$.}
%%\nomenclature{$\shfcoH{X,\shf{I},n}$}{The $n$th cohomology of the sheaf $\shf{I}$ over the space $X$}.
%\nomenclature[\seccfm]{$\Vec_{\grk}$}{Category of Vector Spaces over $\grk$}
%\nomenclature[\seccfm]{$\divisor{p},\divisoridshf{\divisor{p}}$}{A divisor on some ambient space
%along with its ideal sheaf}
%\nomenclature[\seccfm]{$\AJ$}{The Abel Jacobi map taking a divisor $\divisor{p}$
%  supported on a space $X$ to $\AJ: \divisor{p} \mapsto \ringint_{X}
%(-\divisor{p})\divisoridshf{I_{\divisor{p}}}$}
%\nomenclature[\seccfm]{$\IndProSch$}{The category of $\Ind$-$\Pro$-Schemes}
%\nomenclature[\seccoh]{$\grpring{Z}{G}$}{The group ring of $G$}
%\nomenclature[\seccoh]{$\Bun G$}{The moduli space of $G$ bundles.}
%\nomenclature[coh]{$\dualmod{M}$}{The dual module $\Hom{A}(M,A)$ associated to an object $M \in \ModCat{A}$ for some ring $A$}
%\nomenclature{$\LModCat{A}$}{The category of left modules over a ring $A$.}
%\nomenclature{$\RModCat{A}$}{The category of right modules over a ring $A$.}
%\nomenclature[\secindobj]{$\Schemes$}{The category of Schemes}
%\nomenclature[\secgaugetheory]{$\SL_2$}{The reductive affine algebraic group scheme
%  associated to the Special Linear Group of invertible determinant $1$ matrices
%  i.e.\ the fiber $det^{-1}(-1)$ in $\Mat{2}$. Equivalently characterized via a
%  \textit{functor of points} formalism e.g.\ $\grkAlg \rightarrow \Set$
%  represented in $\ZAlg$ by $\grpring{\Z}{\SL_2} \isom
%  \polyring{Z}{4}\ov{\idealgenby{z_1z_3 - z_2z_4 - 1}}$}
%\nomenclature[\secindobj]{$\NoethSch$}{The category of Noetherian Schemes}
%\nomenclature[\secindobj]{$\NoethIntSch$}{The category of Integral Noetherian Schemes}
%\nomenclature[\secindobj]{$\Sch{\grk}$}{The category of Schemes defined over $\grk$}
%\nomenclature[\secindobj]{$\IndSch{\grk}$}{The category of IndSchemes defined over $\grk$}
%\nomenclature[\secindobj]{$\ModCat{A}$}{The abelian category of modules over a
%%%%%%%%%%%%%%%%%%%%%%%%%%
%% %\nomenclature[\secgaugetheory] %%
%%
%%
%%\nomenclature{$\hcoH{G,A,n}$}{blah}
%\nomenclature{$\Gm$}{The multiplicative group as commutative reductive algebraic group.}
%\nomenclature{$\lpgrass$}{The classical loop grassmannian of $G$.}
%\nomenclature{$\glpgrass$}{The \textit{generalized} loop grassmannian of $G$}
%\nomenclature[\seccfm]{$\Flds$}{The Category of Fields}
%\nomenclature[\seccfm]{$\cartdual$}{Cartier duality on a category } % TODO say more}
%\nomenclature{$\Mat{n}\grk$}{$n\times n$ matrices with coefficients in $\grk$}
%\nomenclature{$\RH^*$}{Right derived functor of homology}
%\nomenclature{$X\tateshiftedby{n}$}{The $n$th Tate Shift}
%\nomenclature{$\TateCat$}{The tate category associated to a finite
%  abelian group $H$, defined as the categorical quotient
%  $\mathcal{A}^G_{N_G(\mathcal{A}}$}
%\nomenclature{$\B{G}$}{everyone's favourite quotient category}
%\nomenclature{$\ptmod{G}$}{everyone's favourite quotient category}
\nomenclature{$\DMod (X)$}{The category $\DMod$ of $D$-modules defined on $X$}
\nomenclature{$\IndSt$}{The Category of $\Ind$-Stacks}
\nomenclature{$\IndNSt$}{The Category of $\Ind$-$N$-Stacks}

\printnomenclature

\import{sections/}{todo.tex}
\import{sections/}{notes-on-notes.tex}
\import{sections/}{references.tex}
\import{sections/}{general-conventions.tex}
\import{sections/}{main-examples.tex}
\import{sections/}{reptheory-generalities.tex}
\import{sections/}{reptheory-concrete-examples.tex}
\import{sections/}{affine-liealgebras.tex}
\import{sections/}{unfinished.tex}

\end{document}
