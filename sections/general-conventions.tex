\documentclass[etingof-lie.tex]{subfiles}
\begin{document}
\subsection{General conventions}

We will almost always work over $\mathbb{C}$ in this course. All algebras are
over $\mathbb{C}$ unless specified otherwise. Characteristic $p$ is too
complicated for us, although very interesting. Sometimes we will work over
$\mathbb{R}$, and occasionally even over rings (as auxiliary constructions
require this).

Some remarks on notation:

\begin{itemize}
\item In the following, $\mathbb{N}$ will always denote the set $\left\{
0,1,2,...\right\}  $ (and not $\left\{  1,2,3,...\right\}  $).

\item All rings are required to have a unity (but not necessarily be
commutative). If $R$ is a ring, then all $R$-algebras are required to have a
unity and satisfy $\left(  \lambda a\right)  b=a\left(  \lambda b\right)
=\lambda\left(  ab\right)  $ for all $\lambda\in R$ and all $a$ and $b$ in the
algebra. (Some people call such $R$-algebras \textit{central }$R$%
\textit{-algebras}, but for us this is part of the notion of an $R$-algebra.)

\item When a Lie algebra $\mathfrak{g}$ acts on a vector space $M$, we will
denote the image of an element $m\in M$ under the action of an element
$a\in\mathfrak{g}$ by any of the three notations $am$, $a\cdot m$ and
$a\rightharpoonup m$. (One day, I will probably come to an agreement with
myself and decide which of these notations to use, but for now expect to see
all of them used synonymously in this text. Some authors also use the notation
$a\circ m$ for the image of $m$ under the action of $a$, but we won't use this notation.)

\item If $V$ is a vector space, then the tensor algebra of $V$ will be denoted
by $T\left(  V\right)  $; the symmetric algebra of $V$ will be denoted by
$S\left(  V\right)  $; the exterior algebra of $V$ will be denoted by $\wedge
V$.

\item For every $n\in\mathbb{N}$, we let $S_{n}$ denote the $n$-th symmetric
group (that is, the group of all permutations of the set $\left\{
1,2,\ldots,n\right\}  $). On occasion, the notation $S_{n}$ will denote some
other things as well; we hope that context will suffice to keep these meanings apart.
\end{itemize}

\end{document}
