\documentclass[etingof-lie.tex]{subfiles}
\begin{document}
\subsection{Introduction}

These notes follow a one-semester graduate class by Pavel Etingof at MIT in
the Spring term of 2012. The class was also accompanied by the lecturer's
handwritten notes, downloadable from
\texttt{\href{http://www-math.mit.edu/~etingof/}{\texttt{http://www-math.mit.edu/\symbol{126}%
etingof/}}} .

The goal of these lectures is to discuss the structure and the representation
theory (mainly the latter) of some of the most important infinite-dimensional
Lie algebras.\footnote{It should be noticed that most of the
infinite-dimensional Lie algebras studied in these notes are $\mathbb{Z}%
$-graded and have both their positive and their negative parts
infinite-dimensional. This is in contrast to many Lie algebras appearing in
algebraic combinatorics (such as free Lie algebras over non-graded vector
spaces, and the Lie algebras of primitive elements of many combinatorial Hopf
algebras), which tend to be concentrated in nonnegative degrees. So a better
title for these notes might have been ``Two-sided infinite-dimensional Lie
algebras''.} Occasionally, we are also going to show some connections of this
subject to other fields of mathematics (such as conformal field theory and the
theory of integrable systems).

\begin{verlong}
Also, there are connections to the theory of quantum groups, but we won't get
to them.
\end{verlong}

The prerequisites for reading these notes vary from section to section. We are
going to liberally use linear algebra, the basics of algebra (rings, fields,
formal power series, categories, tensor products, tensor algebras, symmetric
algebras, exterior algebras, etc.) and fundamental notions of Lie algebra
theory. At certain points we will also use some results from the
representation theory of finite-dimensional Lie algebras, as well as some
properties of symmetric polynomials (Schur polynomials in particular) and
representations of associative algebras. Analysis and geometry will appear
very rarely, and mostly to provide intuition or alternative proofs.

The biggest difference between the theory of finite-dimensional Lie algebras
and that of infinite-dimensional ones is that in the finite-dimensional case,
we have a complete picture (we can classify simple Lie algebras and their
finite-dimensional representations, etc.), whereas most existing results for
the infinite-dimensional case are case studies. For example, there are lots
and lots of simple infinite-dimensional Lie algebras and we have no real hope
to classify them; what we can do is study some very specific classes and
families. As far as their representations are concerned, the amount of general
results is also rather scarce, and one mostly studies concrete
families\footnote{Though, to be honest, we are mostly talking about
infinite-dimensional representations here, and these are not very easy to
handle even for finite-dimensional Lie algebras.}.

The main classes of Lie algebras that we will study in this course are:

\textbf{1.} The Heisenberg algebra (aka oscillator algebra) $\mathcal{A}$ and
its Lie subalgebra $\mathcal{A}_{0}$.

\textbf{2.} The Virasoro algebra $\operatorname*{Vir}$.

\textbf{3.} The Lie algebra $\mathfrak{gl}_{\infty}$ and some variations on it
($\overline{\mathfrak{a}_{\infty}}$, $\mathfrak{a}_{\infty}$, $\mathfrak{u}%
_{\infty}$).

\textbf{4.} Kac-Moody algebras (this class contains semisimple Lie algebras
and also affine Lie algebras, which are central extensions of $\mathfrak{g}%
\left[  t,t^{-1}\right]  $ where $\mathfrak{g}$ is simple finite-dimensional).

\end{document}
