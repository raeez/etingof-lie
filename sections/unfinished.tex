\documentclass[etingof-lie.tex]{subfiles}
\begin{document}
\subsection{\textbf{[unfinished]} ...}

[...]
\section{\textbf{[unfinished]} ...}

[...] [747l22.pdf]

KZ equations, consistent (define a flat connection)

$\mathfrak{g}$ simple Lie algebra

$V_{1},V_{2},...,V_{N}$ representations of $\mathfrak{g}$ from Category
$\mathcal{O}$.

$\mathbb{C}_{0}^{N}=\mathbb{C}^{N}\diagdown\left\{  z_{i}=z_{j}\right\}  $

$U\subseteq\mathbb{C}_{0}^{N}$ simply connected open set

$F\left(  z_{1},...,z_{N}\right)  \in\left(  V_{1}\otimes V_{2}\otimes
...\otimes V_{N}\right)  \left[  \nu\right]  $ holomorphic function in
$z_{1},...,z_{N}$ for a fixed weight $\nu$.

$x\in\mathbb{C}$ [or was it $\kappa\in\mathbb{C}$ ?]

$\dfrac{\partial F}{\partial z_{i}}-\overline{h}\sum\limits_{i\neq j}%
\dfrac{\Omega_{i,j}}{z_{i}-z_{j}}F$ where $\Omega_{i,j}:V_{1}\otimes
V_{2}\otimes...\otimes V_{N}\rightarrow V_{1}\otimes V_{2}\otimes...\otimes
V_{N}$

$\Omega\in\left(  S^{2}\mathfrak{g}\right)  ^{\mathfrak{g}}$

Consistent means: setting $\nabla_{i}=\dfrac{\partial}{\partial z_{i}%
}-\overline{h}\sum\limits_{i\neq j}\dfrac{\Omega_{i,j}}{z_{i}-z_{j}}$, we have
$\left[  \nabla_{i},\nabla_{j}\right]  =0$. Consistent systems are known to
have locally unique-and-existent solutions.

Why is this in our course?

The reason is that these equations arise in the representation theory of
affine Lie algebras.

Interpretation of KZ equations in terms of $\widehat{\mathfrak{g}}$:

Consider $L\mathfrak{g}$, $\widehat{\mathfrak{g}}$, $\widetilde{\mathfrak{g}%
}=\widehat{\mathfrak{g}}\rtimes\mathbb{C}d$.

Define Weyl modules:

\begin{definition}
Let $\lambda\in P_{+}$ be a dominant integral weight for a simple
finite-dimensional Lie algebra $\mathfrak{g}$. Let $L_{\lambda}$ be an
irreducible finite-dimensional representation of $\mathfrak{g}$ with highest
weight $\lambda$. Let us extend $L_{\lambda}$ to a $\mathfrak{g}\left[
t\right]  \oplus\mathbb{C}K$-module by making $t\mathfrak{g}\left[  t\right]
$ act by $0$ and $K$ act by some scalar $k$ (that is, $K\mid_{L_{\lambda}%
}=k\cdot\operatorname*{id}$ for some $k\in\mathbb{C}$).

Denote this $\mathfrak{g}\left[  t\right]  \oplus\mathbb{C}K$-module by
$L_{\lambda}^{\left(  k\right)  }$. Then, we define a $\widehat{\mathfrak{g}}%
$-module $V_{\lambda,k}=U\left(  \widehat{\mathfrak{g}}\right)  \otimes
_{U\left(  \mathfrak{g}\left[  t\right]  \oplus\mathbb{C}K\right)  }%
L_{\lambda}^{\left(  k\right)  }$. This module is called a \textit{Weyl
module} for $\widehat{\mathfrak{g}}$ at level $k$.
\end{definition}

By the PBW theorem, we immediately see that $U\left(  \widehat{\mathfrak{g}%
}\right)  \cong U\left(  t^{-1}\mathfrak{g}\left[  t^{-1}\right]  \right)
\otimes U\left(  \mathfrak{g}\left[  t\right]  \oplus\mathbb{C}K\right)  $ and
thus $V_{\lambda,k}\cong U\left(  t^{-1}\mathfrak{g}\left[  t^{-1}\right]
\right)  \otimes L_{\lambda}$ (canonically, but only as vector spaces).

Assuming that $k\neq-h^{\vee}$, we can extend $V_{\lambda,k}$ to
$\widetilde{\mathfrak{g}}$ by letting $d$ act as $-L_{0}$ (from Sugawara construction).

\begin{definition}
If $V$ is a $\mathfrak{g}$-module, then $V\left[  z,z^{-1}\right]  $ is an
$L\mathfrak{g}$-module, and in fact a $\widehat{\mathfrak{g}}$-module where
$K$ acts by $0$. It extends to $\widetilde{\mathfrak{g}}$ by setting
$d=z\dfrac{\partial}{\partial z}$.

More generally: Can set $d\left(  vz^{n}\right)  =\left(  n-\Delta\right)
vz^{n}$ for any fixed $\Delta\in\mathbb{C}$.

Call this module $z^{-\Delta}V\left[  z,z^{-1}\right]  $.
\end{definition}

\begin{lemma}
If $k\notin\mathbb{Q}$, then $V_{\lambda,k}$ is irreducible.
\end{lemma}

\textit{Proof of Lemma.} Assume $V_{\lambda,k}$ is reducible. This
$V_{\lambda,k}$ is a highest-weight module. So, it must have a singular vector
in degree $\ell>0$. Let $C$ be the Casimir for $\widetilde{\mathfrak{g}}$. We
know $C=L_{0}-\deg$ (where $\deg$ returns the positive degree).

Assume that $w$ (our singular vector) lives in an irr. repr. of $\mathfrak{g}%
$. Singular vector means $a\left(  m\right)  w=0$ for all $m>0$. Here
$a\left(  m\right)  $ means $at^{m}$.

$C\mid_{V_{\lambda,k}}=\dfrac{\left(  \lambda,\lambda+2\rho\right)  }{2\left(
k+h^{\vee}\right)  }$

$Cw=\left(  \dfrac{\left(  \mu,\mu+2\rho\right)  }{2\left(  k+h^{\vee}\right)
}-\ell\right)  w$

$L_{0}=\dfrac{1}{2\left(  k+h^{\vee}\right)  }\sum_{i\in\mathbb{Z}}\sum_{a\in
B}:a\left(  i\right)  a\left(  -i\right)  :\ =\dfrac{1}{2\left(  k+h^{\vee
}\right)  }\left(  \sum_{a\in B}a\left(  0\right)  ^{2}+2\sum_{a\in B}%
\sum_{m\geq1}a\left(  -m\right)  a\left(  m\right)  \right)  $ where $a\left(
m\right)  =at^{m}$.

$\Longrightarrow$ $\underbrace{\left(  \lambda,\lambda+2\rho\right)  =\left(
\mu,\mu+2\rho\right)  }_{\in\mathbb{Z}}-2\ell\left(  k+h^{\vee}\right)  $
$\Longrightarrow$ $k=-h^{\vee}+\dfrac{\left(  \lambda,\lambda+2\rho\right)
-\left(  \mu,\mu+2\rho\right)  }{2\ell}\in\mathbb{Q}$. $\Longrightarrow$ contradiction.

\begin{corollary}
If $k\notin\mathbb{Q}$, then $V_{\lambda,k}^{\ast}$ (restricted dual) is
$U\left(  \widehat{\mathfrak{g}}\right)  \otimes_{U\left(  \mathfrak{g}\left[
t^{-1}\right]  \oplus\mathbb{C}K\right)  }L_{\lambda}^{\ast\left(  -k\right)
}$. (Here, $L_{\lambda}^{\ast\left(  -k\right)  }$ means $L_{\lambda}^{\ast}$
with $K$ acting as $-k$.)
\end{corollary}

\textit{Proof of Corollary.} From Frobenius reciprocity, we have a
homomorphism $\varphi:U\left(  \widehat{\mathfrak{g}}\right)  \otimes
_{U\left(  \mathfrak{g}\left[  t^{-1}\right]  \oplus\mathbb{C}K\right)
}L_{\lambda}^{\ast\left(  -k\right)  }\rightarrow V_{\lambda,k}^{\ast}$ which
is $\operatorname*{id}$ in degree $0$. In fact, Frobenius reciprocity tells us
that%
\[
\operatorname*{Hom}\nolimits_{\widehat{\mathfrak{g}}}\left(  U\left(
\widehat{\mathfrak{g}}\right)  \otimes_{U\left(  \mathfrak{g}\left[
t^{-1}\right]  \oplus\mathbb{C}K\right)  }L_{\lambda}^{\ast\left(  -k\right)
},M\right)  \cong\operatorname*{Hom}\nolimits_{\mathfrak{g}\left[
t^{-1}\right]  \oplus\mathbb{C}K}\left(  L_{\lambda}^{\ast\left(  -k\right)
},M\right)  ,
\]
which, in the case $M=V_{\lambda,k}^{\ast}$, becomes [...].

Because $V_{\lambda,k}$ is irreducible (here we are using $k\notin\mathbb{Q}%
$), $V_{\lambda,k}^{\ast}$ is irreducible as well, this homomorphism $\varphi$
is surjective. This $\varphi$ also preserves grading, and the characters are
equal. $\Longrightarrow$ $\varphi$ is an isomorphism.

\begin{corollary}
$\operatorname*{Hom}\nolimits_{\widetilde{\mathfrak{g}}}\left(  V_{\lambda
,k}\otimes V_{\nu,k}^{\ast},z^{-\Delta}V\left[  z,z^{-1}\right]  \right)
\cong\operatorname*{Hom}\nolimits_{\mathfrak{g}}\left(  L_{\lambda}\otimes
L_{\nu}^{\ast},V\right)  $ if $\Delta=\Delta\left(  \lambda\right)
-\Delta\left(  \nu\right)  $.
\end{corollary}

\textit{Proof of Corollary.} Frobenius reciprocity as for the previous
corollary. (Skip.)

[...]

We now cite a classical theorem on ODEs.

\begin{theorem}
Let $N\in\mathbb{N}$. Let $A\left(  z\right)  =A_{0}+A_{1}z+A_{2}z^{2}+...$ be
a holomorphic function on $\left\{  z\in\mathbb{C}\ \mid\ \left\vert
z\right\vert <1\right\}  $ with values in $\operatorname*{M}\nolimits_{N}%
\left(  \mathbb{C}\right)  $. Assume that for any eigenvalues $\lambda$ and
$\mu$ of $A_{0}$ such that $\lambda\neq\mu$, one has $\lambda-\mu
\notin\mathbb{Z}$. Then, the ODE $z\dfrac{dF}{dz}=A\left(  z\right)  F$
(which, of course, is equivalent to $\dfrac{dF}{dz}=\dfrac{A\left(  z\right)
}{z}F$) has a matrix solution of the form $F\left(  z\right)  =\left(
1+B_{1}z+B_{2}z^{2}+...\right)  z^{A_{0}}$ such that the power series
$1+B_{1}z+B_{2}z^{2}+...$ converges for $\left\vert z\right\vert <1$. Here,
$z^{A_{0}}$ means $\exp\left(  A_{0}\log z\right)  $ (on $\mathbb{C}%
\diagdown\mathbb{R}_{\leq0}$).
\end{theorem}

\begin{remark}
This is a development of the following basic theorem: If we are given an ODE
$\dfrac{dF}{dz}=C\left(  z\right)  F$ with $C\left(  z\right)  $ holomorphic,
then there exists a holomorphic $F$ satisfying this equation and having the
form $F=1+O\left(  z\right)  $ (the so-called fundamental equation).
\end{remark}

\textit{Proof of Theorem.} Plug in the solution $F\left(  z\right)  $ in the
above formula:%
\[
\left(  \sum\limits_{n\geq1}nB_{n}z^{n}\right)  z^{A_{0}}+\left(
1+\sum\limits_{n\geq1}B_{n}z^{n}\right)  A_{0}z^{A_{0}}=\left(  A_{0}%
+A_{1}z+A_{2}z^{2}+...\right)  \left(  1+B_{1}z+B_{2}z^{2}+...\right)
z^{A_{0}}.
\]
Cancel $z^{A_{0}}$ from this to obtain%
\[
\sum\limits_{n\geq1}nB_{n}z^{n}+\left(  1+\sum\limits_{n\geq1}B_{n}%
z^{n}\right)  A_{0}=\left(  A_{0}+A_{1}z+A_{2}z^{2}+...\right)  \left(
1+B_{1}z+B_{2}z^{2}+...\right)  .
\]
This is the system of recursive equations%
\[
nB_{n}-A_{0}B_{n}+B_{n}A_{0}=A_{1}B_{n-1}+A_{2}B_{n-2}+...+A_{n-1}B_{1}%
+A_{n}.
\]
This rewrites as%
\[
\left(  n-\operatorname*{ad}A_{0}\right)  \left(  B_{n}\right)  =A_{1}%
B_{n-1}+A_{2}B_{n-2}+...+A_{n-1}B_{1}+A_{n}.
\]
The operator $n-\operatorname*{ad}A_{0}:\operatorname*{M}\nolimits_{N}\left(
\mathbb{C}\right)  \rightarrow\operatorname*{M}\nolimits_{N}\left(
\mathbb{C}\right)  $ is invertible (because eigenvalues of this operator are
$n-\left(  \lambda-\mu\right)  $ for $\lambda$ and $\mu$ being eigenvalues of
$A_{0}$, and because of the condition that for any eigenvalues $\lambda$ and
$\mu$ of $A_{0}$ such that $\lambda\neq\mu$, one has $\lambda-\mu
\notin\mathbb{Z}$). Hence, we can use the above equation to recursively
compute $B_{n}$ for all $n$.

This implies that a solution in the formal sense exists.

We also need to estimate radius of convergence. [...]

The following generalizes our theorem to several variables:

\begin{theorem}
Let $m\in\mathbb{N}$ and $N\in\mathbb{N}$. For every $i\in\left\{
1,2,...,m\right\}  $, let $A_{i}\left(  \xi_{1},\xi_{2},...,\xi_{m}\right)  $
be a holomorphic on $\left\{  \left(  \xi_{1},\xi_{2},...,\xi_{m}\right)
\ \mid\ \left\vert \xi_{j}\right\vert <1\text{ for all }j\right\}  $ with
values in $\operatorname*{M}\nolimits_{N}\left(  \mathbb{C}\right)  $.
Consider the system of differential equations $\xi_{i}\dfrac{dF}{d\xi_{i}%
}=A_{i}\left(  \xi\right)  F$ for all $i\in\left\{  1,2,...,m\right\}  $ on a
single function $F:\mathbb{C}^{m}\rightarrow\operatorname*{M}\nolimits_{N}%
\left(  \mathbb{C}\right)  $. Assume
\[
\left[  \xi_{i}\dfrac{d}{d\xi_{i}}-A_{i},\xi_{j}\dfrac{d}{d\xi_{j}}%
-A_{j}\right]  =0\ \ \ \ \ \ \ \ \ \ \text{for all }i,j\in\left\{
1,2,...,m\right\}
\]
(this is called a \textit{consistency condition}, aka a zero curvature
equation). Then, $\left[  A_{i}\left(  0\right)  ,A_{j}\left(  0\right)
\right]  =0$ for all $i,j\in\left\{  1,2,...,m\right\}  $, and thus the
matrices $A_{i}\left(  0\right)  $ for all $i$ can be simultaneously
trigonalized. Under this trigonalization, let $\lambda_{i,1}$, $\lambda_{i,2}%
$, $...$, $\lambda_{i,N}$ be the diagonal entries of $A_{i}\left(  0\right)  $.

Assume that the condition
\[
\left(  \lambda_{1,k}-\lambda_{1,\ell},\lambda_{2,k}-\lambda_{2,\ell
},...,\lambda_{m,k}-\lambda_{m,\ell}\right)  \notin\mathbb{Z}^{m}\diagdown0
\]
holds for all $k$ and $\ell$. [...]
\end{theorem}



\end{document}
